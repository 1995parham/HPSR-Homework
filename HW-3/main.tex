\documentclass[paper=a4, fontsize=11pt]{article}

%----------------------------------------------------------------------------------------
%	PACKAGES AND OTHER DOCUMENT CONFIGURATIONS
%----------------------------------------------------------------------------------------

\usepackage{amsmath,amsfonts,amsthm} % Math packages
\usepackage{sectsty} % Allows customizing section commands
\allsectionsfont{\centering \normalfont\scshape} % Make all sections centered, the default font and small caps

\usepackage{fancyhdr} % Custom headers and footers
\pagestyle{fancyplain} % Makes all pages in the document conform to the custom headers and footers
\fancyhead{} % No page header - if you want one, create it in the same way as the footers below
\fancyfoot[L]{} % Empty left footer
\fancyfoot[C]{} % Empty center footer
\fancyfoot[R]{\thepage} % Page numbering for right footer
\renewcommand{\headrulewidth}{0pt} % Remove header underlines
\renewcommand{\footrulewidth}{0pt} % Remove footer underlines
\setlength{\headheight}{13.6pt} % Customize the height of the header

\numberwithin{equation}{section} % Number equations within sections (i.e. 1.1, 1.2, 2.1, 2.2 instead of 1, 2, 3, 4)
\numberwithin{figure}{section} % Number figures within sections (i.e. 1.1, 1.2, 2.1, 2.2 instead of 1, 2, 3, 4)
\numberwithin{table}{section} % Number tables within sections (i.e. 1.1, 1.2, 2.1, 2.2 instead of 1, 2, 3, 4)

%\setlength\parindent{0pt} % Removes all indentation from paragraphs - comment this line for an assignment with lots of text
\setlength{\parskip}{1em}
\renewcommand{\baselinestretch}{1.5}

\usepackage{tikz-qtree}
\usepackage{lscape}

\usepackage{graphicx}
\graphicspath{ {images/} }

\usepackage{xepersian}
\settextfont[Path=fonts/]{Vazir.ttf}
%\setlatintextfont{Times New Roman}

%----------------------------------------------------------------------------------------
%	TITLE SECTION
%----------------------------------------------------------------------------------------

\newcommand{\horrule}[1]{\rule{\linewidth}{#1}} % Create horizontal rule command with 1 argument of height

\title{
\normalfont\normalsize
\includegraphics[scale=0.1]{aut}
\hspace{5cm}
\includegraphics[scale=0.1]{ceit} \\
\textsc دانشگاه صنعتی امیرکبیر \\
\textsc دانشکده مهندسی کامپیوتر و فناوری اطلاعات
\horrule{0.5pt} \\ [0.4cm] % Thin top horizontal rule
\huge معماری سوئیچ و روترهای با کارآیی بالا \\ % The assignment title
\huge تمرین سوم \\ % The assignment title
\horrule{2pt} \\ [0.5cm] % Thick bottom horizontal rule
}

\author{پرهام الوانی}

\date{\normalsize\today} % Today's date or a custom date

\begin{document}

\maketitle % Print the title

\section{سوال اول}
\par
باس باید بتواند اگر همه پورت‌های ورودی داده داشتند آن‌ها را منتفل کند پس می‌بایست به اندازه‌ی مجموع نرخ همه‌ی پورت‌های ورودی
ظرفیت داشته باشد.
\begin{align}\begin{split}
    BW_{bus} = 24 * 10 = 240 Gb/s
\end{split}\end{align}

\section{سوال دوم}
\par
برای سوئیچ‌هایی با معماری \lr{shared-memory} می‌دانیم:
\begin{align}\begin{split}
    2 * N_r \le \frac{T_{cell}}{T_{mem}}
\end{split}\end{align}

از آنجایی که شبکه‌ای که این سوئیچ به آن متصل است شبکه‌ی \lr{ATM}
می‌باشد پس اندازه‌ی بسته‌ها در آن ۵۳ بایت می‌باشد.
\begin{align}\begin{split}
    T_{cell} = \frac{L_{cell}}{throughput}\\
    & = \frac{53 * 8}{125 * 10^3} = 3.392 ms
\end{split}\end{align}

و در نهایت خواهیم داشت:
\begin{align}\begin{split}
    N_r = \frac{T_{cell}}{T_{mem}} = \frac{3.392ms}{16ns}\\
    & = 212
\end{split}\end{align}

\section{سوال سوم}
\begin{itemize}
    \item \lr{Throughput}:
    گذردهی یک سوئیچ بنابر تعریف نسبت میانگین تجمعی پورت‌های خروجی به میانگین تجمعی پورت‌های ورودی می‌باشد.
    گذردهی یک عدد مثبت و کمتر از یک است.
    \item \lr{Speedup}:
    \lr{Speedup}، $k$
    به این معناست که نرخ \lr{forwarding}
    $k$ برابر نرخ خط ورودی می‌باشد.
    \item \lr{Blocking}:
    \lr{Blocking} زمانی در یک سوئیچ وجود دارد که
    یک پورت خروجی آزاد وجود دارد که یک پورت ورودی آزاد می‌خواهد از طریق آن ارسال کند ولی این امر ممکن نیست.
\end{itemize}

\section{سوال چهارم}
فرض کنید می‌خواهیم ورودی $a$ را به $b$
وصل کنیم ولی تمام $n-1$ ووردی دیگر
به تمام $n-1$ خروجی دیگر از طریق سوئیچ‌های میانی متصل شده است،
بنابراین $2n-2$ سوئیچ میانی استفاده شده است و
برای اتصال $a$ به $b$ به یک سوئیچ میانی دیگر نیاز خواهیم داشت:

$m \ge 2(n - 1) + 1 = 2n - 1$

\section{سوال پنجم}

\section{سوال ششم}

\end{document}